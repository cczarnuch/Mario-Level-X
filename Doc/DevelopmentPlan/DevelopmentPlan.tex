\documentclass{article}

\usepackage{booktabs}
\usepackage{tabularx}

\title{SE 3XA3: Development Plan\\ Mario Level X}

\author{Team 210, Team Name
		\\ Ahmad Gharib, ghariba
		\\ Connor Czarnuch, czarnucc
		\\ Edward Liu, liuz150
}

\date{}

\begin{document}

\begin{table}[hp]
\caption{Revision History} \label{TblRevisionHistory}
\begin{tabularx}{\textwidth}{llX}
\toprule
\textbf{Date} & \textbf{Developer(s)} & \textbf{Change}\\
\midrule
2020-01-30 & All members & Work Division\\
2020-01-31 & All members & Complete first draft of Development Plan\\
\bottomrule
\end{tabularx}
\end{table}

\newpage

\maketitle

This is an ongoing document that will track our development plan for our open-source project development.

\section{Team Meeting Plan}
Whenever any member of the group feels the need to meet for the project, they are free to arrange a time with the group in order to discuss the project. We aim to meet or discuss our progress at least once a week. All meetings minutes are to be tracked as well, in order to minimize conflicts between group members.
\section{Team Communication Plan}
We will mostly keep in touch by communicating over Discord messenger. This is so we can easily contact each other as we all use Discord frequently.
\section{Team Member Roles}
Ahmad Gharib: LaTeX, Programmer \\
Connor Czarnuch: Git, Programmer \\
Edward Liu: PyGame, Programmer
\section{Git Workflow Plan}
-connor
\section{Proof of Concept Demonstration Plan}
-connor
\section{Technology}
-connor
\section{Coding Style}
The coding style that we have chosen is pyguide by google https://google.github.io/styleguide/pyguide.html. The current project conforms to this style and we would like to maintain the same style as the original.
\section{Project Schedule}
Project schedule is recorded using GantProject. The ``.gan'' file can be found at https://gitlab.cas.mcmaster.ca/czarnucc/Mario-Level-1/tree/master/ProjectSchedule. Gant chart will be updated throughout the project.

\section{Project Review}
Currently, our customized modules has integrated with the exisitng code well. However, we have not updated ourselves with pygame so we cannot perform advanced functionalities yet. We predict that the level editor ui will be more difficult since it requires more knowledge of pygame.
\end{document}