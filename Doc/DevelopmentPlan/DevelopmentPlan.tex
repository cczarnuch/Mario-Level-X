\documentclass{article}

\usepackage{booktabs}
\usepackage{tabularx}

\title{SE 3XA3: Development Plan\\ Mario Level X}

\author{Team 210, Team Name
		\\ Ahmad Gharib, ghariba
		\\ Connor Czarnuch, czarnucc
		\\ Edward Liu, liuz150
}

\date{}

\begin{document}

\begin{table}[hp]
\caption{Revision History} \label{TblRevisionHistory}
\begin{tabularx}{\textwidth}{llX}
\toprule
\textbf{Date} & \textbf{Developer(s)} & \textbf{Change}\\
\midrule
2020-01-30 & All members & Work Division\\
2020-01-31 & All members & Complete first draft of Development Plan\\
\bottomrule
\end{tabularx}
\end{table}

\newpage

\maketitle

This is an ongoing document that will track our development plan for our open-source project development.

\section{Team Meeting Plan}
Whenever any member of the group feels the need to meet for the project, they are free to arrange a time with the group in order to discuss the project. We aim to meet or discuss our progress at least once a week. All meetings minutes are to be tracked as well, in order to minimize conflicts between group members.
\section{Team Communication Plan}
We will mostly keep in touch by communicating over Discord messenger. This is so we can easily contact each other as we all use Discord frequently.
\section{Team Member Roles}
Ahmad Gharib: LaTeX, Programmer \\
Connor Czarnuch: Git, Programmer \\
Edward Liu: PyGame, Programmer
\section{Git Workflow Plan}
For our project, we will use a single master branch and multiple story branches. Each story branch will allow a developer to work on a feature in their own branch so that there will be no conflicts of work if multiple developers are working on the same file. Once the work is done in the story branch, the developer will push and merge to the master branch using the gitlab online interface. Since master is a protected branch, developers are unable to merge directly on the command line so using a protected branch makes it easier to avoid merging errors. After the work on a story is completed, the branch will be deleted. At various points during the project, the master branch will be tagged, allowing the Professor and TAs are able to mark our work.
\section{Proof of Concept Demonstration Plan}
The first demonstration that we are able to make is the original game working after we re-factor the original codebase to be more modular. After this is done and the work on our added features is completed then we can prove that our added code works in conjunction with the original game.
\section{Technology}
The main technology that this program uses is pygame, a python library that allows a developer to import graphics into their program to create a game.
\section{Coding Style}
The coding style that we have chosen is pyguide by google \\https://google.github.io/styleguide/pyguide.html. The current project conforms to this style and we would like to maintain the same style as the original.
\section{Project Schedule}
Project schedule is recorded using GantProject. The ``.gan'' file can be found at https://gitlab.cas.mcmaster.ca/czarnucc/Mario-Level-1/tree/master/ProjectSchedule. Gant chart will be updated throughout the project.

\section{Project Review}
Currently, our customized modules has integrated with the exisitng code well. However, we have not updated ourselves with pygame so we cannot perform advanced functionalities yet. We predict that the level editor ui will be more difficult since it requires more knowledge of pygame.
\end{document}